\documentclass[14pt,a4paper]{report}

\newcommand{\HRule}{\rule{\linewidth}{0.5mm}}
\makeatletter\@addtoreset{section}{part}\makeatother

\renewcommand\thesection{\arabic{part}.\arabic{section}}
\renewcommand\thesubsection{\thesection.\arabic{subsection}}

\usepackage[hidelinks]{hyperref}
\usepackage{color}
\usepackage{textcomp}
\usepackage{listings}
\usepackage{appendix}
\hypersetup{
    linktoc=all     
}
\definecolor{dkgreen}{rgb}{0,0.6,0}
\definecolor{gray}{rgb}{0.5,0.5,0.5}
\definecolor{mauve}{rgb}{0.58,0,0.82}


\lstset{frame=single,
    language=go,
    aboveskip=3mm,
    belowskip=3mm,
    showstringspaces=false,
    columns=flexible,
    basicstyle={\small\ttfamily},
    numbers=none,
    keywordstyle=\color{blue},
    commentstyle=\color{dkgreen},
    stringstyle=\color{mauve},
    breaklines=true,
    upquote=true,
    breakatwhitespace=true
    tabsize=4,
    morekeywords={bool},
    rulecolor=\color{black}, 
    identifierstyle=\color{black}
}

\begin{document}
\title{
    \textsc{GUC - Media Engineering and Technology Department} \\
    AI Project 1 - 2048
}
\begin{titlepage}
\begin{center}
\textsc{\large German University in Cairo}\\[0.4cm]
\textsc{\large Media Engineering and Technolegy}\\[0.4cm]
\HRule \\[0.4cm]
\textsc{\Large AI Project 1 : 2048}\\[0.4cm]

\HRule \\[1.5cm]

\noindent
\begin{minipage}{0.4\textwidth}
\begin{flushleft} \large
Ahmed H.~Ismail \\
Kareem Ali \\
Muhammed Hassan \\
\end{flushleft}
\end{minipage}
\begin{minipage}{0.4\textwidth}
\begin{flushright} \large
16-4477 \\
22-2213 \\
22-3137 \\
\end{flushright}
\end{minipage} 

\vfill

{\large \today}


\end{center}
\end{titlepage}

\renewcommand{\abstractname}{Brief}
\begin{abstract}
This paper explores a solution to a variant of the famous 2048 game, where new tiles
appear at the corners only instead of a random free tile. Our goal is to formulate a plan
where we reach our goal, a certain power of two \emph{M} between where $1 <= M >=2$. 
Furthermore we associate optimality with minimal score. Thus the optimal path to reach a goal
\emph{M} is the one with the lowest score.

We describe a general search problem that can represent any problem along with a general search algorithm that solves the problem using different search techniques.

We will explore different informed and uniformed search strategies. We describe an implementation of each strategy using Google's Golang. We compare their performance and limitations.


\end{abstract}

\tableofcontents

\part{Models}

\section{Search-tree node}
A single node in a search tree represents a game state. It stores the state's 
grid, maximum value, parent node, operator applied, accumulative cost, depth and
heuristic value. 



\subsection{Node interface}
The node interface represents the abstract search tree node it is declared
in \begin{verbatim}types.go\end{verbatim}
\begin{lstlisting}
type Node interface {
    get_parent() Node
    // Returns the operator used to reach this node
    get_operator() int
    // Returns path depth with root having a depth of 0
    get_depth() int
    // Returns the path cost, which represents the score
    get_path_cost() interface
    // Returns the path from root to this node (series of actions)
    get_path() Path
    // Returns true if it's possible to apply operator on this node
    can_apply(operator int) bool
    // Applies operator and returns the resulting node
    apply(operator int) Node
}
\end{lstlisting}

\subsection{N2048 struct}
The struct N2048 represents the search-tree node, it is declared in 
\begin{verbatim}n2048.go\end{verbatim} it implements the Node interface 
\begin{lstlisting}
type N2048 struct {
    board      Grid   // The grid
    max        int    // maximum value in board
    parent     *N2048 // nil if root
    operator   int    // operator applied on parent to reach this node
    path_cost  int    // cost from initial state to here
    depth      int    // distance from initial state
    score      int    // heuristic score
    score_flag bool   //score flag is set
}
\end{lstlisting}
\vfill


\section{Search problem}
An abstract problem should be able to do three things
\begin{itemize}
\item Provide the initial state; root of search tree
\item Be able to tell if a search tree node satisfies the goal constraints
\item Expand a node; get it's children
\end{itemize}
We use the Problem interface to model an abstract problem

\subsection{Problem interface}
The Problem interface is declared in \verb+ types.go+ 
\begin{lstlisting}
type Problem interface {
    initial_state() Node
    goal_test(n Node) bool
    expand(n Node) []Node
}
\end{lstlisting}
\subsection{P2048 struct}
The 2048 problem is represented by the P2048 struct, which implements the Problem interface.
The P2048 struct is declared and implemented in \verb+ p2048.go+
\begin{lstlisting}
type P2048 struct {
    goal int
    grid *Grid
}
\end{lstlisting}
\vfill

\section{2048 problem}
    
\part{Main Functions}
\section{GenGrid}
The function \verb+ GenGrid+ generates a \verb+ 4x4+ grid, with two '2' tiles placed randomly.
It is defined in the \verb+ main.go+ file. It is implemented as required by the project description.
\begin{lstlisting}
func GenGrid() Grid {
    var grid Grid = Grid(0)
    gobal_hash = make(map[Grid]bool)
    rand.Seed(time.Now().UTC().Unix())
    rand.Seed(42)
    r1, c1, r2, c2 := rand.Intn(4), rand.Intn(4), rand.Intn(4), rand.Intn(4)

    for (r1 == r2) && (c2 == c1) {
        c1 = rand.Intn(4)
    }
    grid = grid.grid_ins(r1, c1, 2)
    grid = grid.grid_ins(r2, c2, 2)
    return grid
}
\end{lstlisting}
\section{Search}
As per the project description the function \verb+ Search+ uses a search strategy to 
formulate a plan. It is defined in the \verb+ main.go+ file.
\begin{lstlisting}
func Search(grid Grid, M int, strategy int, visualize bool) (p Path, cost int, nodes uint64) {
    problem := P2048{M, grid}
    global_problem = &problem
    var (
        target         Node
        success        bool
        nodes_expanded uint64
        goal_path      Path
        path_cost      int
    )
    if strategy == ID {
        target, success, nodes_expanded = iterative_deepening_search(&problem)

    } else {
        quing_func := get_quing_func(strategy)
        target, success, nodes_expanded = general_search(&problem, quing_func)
    }

    if success {
        // reached goal state
        goal_path, path_cost = target.get_path(), target.get_path_cost()

    } else {
        // Failed to reach goal state
        goal_path, path_cost = Path{}, 0
    }

    if visualize {
        display(goal_path.encode())
    }

    return goal_path, path_cost, nodes_expanded
}
\end{lstlisting}

The strategy parameter is one of the following constants.
\begin{enumerate}
\item BF
\item DF
\item ID
\item GR1
\item GR2
\item AS1
\item AS2
\end{enumerate}
Which forces search to use the selected strategy as per the project description.
\section{General Search}
The general search function is implemented in \verb+ search.go+
\begin{lstlisting}
func general_search(p Problem, quing_fun Strategy) (Node, bool, uint64) {
    nodes := make([]Node, 0, 10)             // Make a queue
    nodes = append(nodes, p.initial_state()) // queue initial state
    expanded_nodes := uint64(0)
    for {
        if len(nodes) == 0 {
            return nil, false, expanded_nodes
        } else {
            // Select first node
            node := nodes[0]
            if p.goal_test(node) {
                return node, true, expanded_nodes
            } else {
                expanded_nodes++
                nodes = quing_fun(&nodes, p.expand(node))
                nodes = nodes[:len(nodes)-1]
            }
        }
    }
}
\end{lstlisting}
It returns a three tuple containing
\begin{enumerate}
\item The goal node or nil if not found
\item A boolean indicating success
\item The number of nodes expanded
\end{enumerate}
\section{Iterative Deepening Search}
Iterative deepening uses the general search function with a queuing function that discards
nodes beyond a certain depth. It is implemented in \verb+ main.go +
\begin{lstlisting}
func iterative_deepening_search(p Problem) (Node, bool, uint64) {

    total_expanded_nodes := uint64(0)
    for limit := uint64(0); limit < math.MaxUint64; limit++ {
        quing_fun := depth_limited_search(limit)
        target, success, expanded_nodes := general_search(p, quing_fun)
        total_expanded_nodes += expanded_nodes
        if success {
            return target, success, total_expanded_nodes
        }

    }

    return nil, false, total_expanded_nodes
}
\end{lstlisting}

It returns the same tuple as the general search function.



\part{Search Algorithms}
\section{Breadth First Search}
When breadth first is selected by the Search function, the general search function is called
with \verb+ enqueue_at_end + as the queuing function which can be found in the \verb+ strategy.go+ file.
\begin{lstlisting}
func enqueue_at_end(nodes *[]Node, children []Node) []Node {
    *nodes = (*nodes)[1:]
    return append(*nodes, children...)
}
\end{lstlisting}
It simply adds the children in Left, Right, Down, Up order.
\section{Depth First Search}
When depth first is selected by the Search function, the general search function is called
with \verb+ enqueue_at_front+ as the queuing function which can be found in the \verb+ strategy.go+ file.
\begin{lstlisting}
func enqueue_at_front(nodes *[]Node, children []Node) []Node {
    *nodes = (*nodes)[1:]
    return append(children, *nodes...)
}
\end{lstlisting}
It simply adds the children in Left, Right, Down, Up order.
\section{Iterative Deepening Search}
A single iteration of iterative deepening search is accomplished by calling general search,
with a special queuing function generated by \verb+depth_limited_search+ function which can be 
found in the \verb+ strategy.go+ file.
\begin{lstlisting}
func depth_limited_search(limit uint64) Strategy {
    return func(nodes *[]Node, children []Node) []Node {
        *nodes = (*nodes)[1:]
        if uint64(children[0].get_depth()) > limit {
            return *nodes
        } else {
            return append(children, *nodes...)
        }
    }
} 
\end{lstlisting}
It is exactly the same as \verb+ enqueue_at_front+ with the exception that any node
having a depth greater than \verb+ limit+ is discarded.
\section{Greedy Search}
When greedy search is selected by the Search function, the general search function is called
with the queuing function returned by \verb+best_fit_enqueue+ which can be found in 
\verb+strategy.go+. \verb+best_fit_enqueue+ returns a queuing function that uses a heap for node 
storage and a heuristic function to generate the key.
\begin{lstlisting}
func best_fit_enqueue(h Heuristic) Strategy {
    return func(nodes *[]Node, children []Node) []Node {
        heap_down(nodes, h)
        for i := 0; i < len(children); i++ {
            heap_up(nodes, children[i], h)
        }
        return *nodes
    }
}
\end{lstlisting}

Where $Heuristic \in \{h1, h2\}$ 

\section{A* Search}
\emph{A*} search works the same way as \emph{Greedy} search, the only difference is that
$Heuristic \in \{h3, h4\}$ which include \emph{g(n)} so technicly \emph{h3} and \emph{h4} are 
\emph{f(n)}s not \emph{h(n)}s 
\part{Heuristics}
\section{h1}
\section{h2}
\section{h3}
\section{h4}


\appendix
\appendixpage
\addappheadtotoc
\section{Setting up Go enviroment}
The project is implemented using \emph{Go} \verb+1.3.1+. \emph{Go} is a cross platform 
language created by \emph{Google}. You can install \emph{Go} and setup it's enviroment
by following the instructions on \url{https://golang.org/doc/install}.
\section{Dependecies}
First you will need to install the \verb+assert+ package to your \verb+gopath+.
and the \verb+open+ package.
\begin{lstlisting}[language=bash]
go get "github.com/stretchr/testify/assert"
go get "github.com/skratchdot/open-golang/open"
\end{lstlisting}


\end{document}